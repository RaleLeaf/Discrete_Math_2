\documentclass{article}
\usepackage{amsmath, amssymb}

\begin{document}

\begin{flushleft}

\textbf{Francis Rale Abrenica   |   BSCS - 1   |    Discrete Structures II}
\newline

\textbf{1. Recall the definition of a rational number, denoted as $\mathbb{Q}$. Prove that Euler's number $e = \sum_{k=0}^\infty \frac{1}{k!} \notin \mathbb{Q}$. A factorial is defined as $k! = (k)(k-1)(k-2)(k-3)..., \forall k \in \mathbb{Z}^+$, note that $0! = 1$. Furthermore, a sum notation $\sum_{k=0}^\infty k = 0+ 1 + 2 + 3 +....+...$.}
\newline\newline
\textbf{ANSWER:}

\textbf{Step 1:}

To demonstrate that Euler's constant $e$ is irrational, we'll use proof by contradiction.

Let's assume that $e$ is rational, meaning it can be represented as the fraction of two integers $p$ and $q$, where $q \neq 0$. Hence, $e = \frac{p}{q}$.

Recall the definition of $e$: 

$ e = \frac{1}{0!} + \frac{1}{1!} + \frac{1}{2!} + \frac{1}{3!} + \ldots $
\newline\newline
\textbf{Step 2:}

Consider the partial sum of the series:

$ S_n = \frac{1}{0!} + \frac{1}{1!} + \frac{1}{2!} + \frac{1}{3!} + \ldots + \frac{1}{n!} $

for some positive integer $n$. As each term in the series is positive, $S_n$ increases and is bounded above by $e$, thus converging to $e$ as $n$ approaches infinity.
\newline\newline
\textbf{Step 3:}

Consider the expression:

$ q! \cdot S_n = q! \cdot \left(\frac{1}{0!} + \frac{1}{1!} + \frac{1}{2!} + \frac{1}{3!} + \ldots + \frac{1}{n!}\right) $
\newline\newline
\textbf{Step 4:}

Expand:

$ q! \cdot S_n = q! + \frac{q!}{1!} + \frac{q!}{2!} + \frac{q!}{3!} + \ldots + \frac{q!}{n!} $
\newline\newline
\textbf{Step 5:}

Every term in this expression is an integer. Since $S_n$ converges to $e$, $q! \cdot S_n$ should be very close to $q! \cdot e = p$. This suggests that $p - q! \cdot S_n$ should be a very small positive integer.

$ p - q! \cdot S_n = p - (q! + \frac{q!}{1!} + \frac{q!}{2!} + \frac{q!}{3!} + \ldots + \frac{q!}{n!}) $
\newline\newline
\textbf{Step 6:}

Every number enclosed in the brackets is a whole number, and every number within the sum is a fraction with a denominator larger than 1. Therefore, the expression $p - q! \cdot S_n$ has to be a positive whole number that isn't zero.

Nevertheless, this conflicts with the assertion that $e = \frac{p}{q}$ because $q! \cdot S_n$ should closely approach $p$ but does not match it precisely, resulting in a contradiction.
\newline\newline
\textbf{Consequently, our presumption that $e$ is rational must be mistaken, thus establishing that $e$ is irrational.}
\newline\newline
\textbf{2. Prove Minkowski's Inequality for sums, $\forall \ p>1, (a_k, b_k)>0$:}

\begin{gather}
\begin{bmatrix}
\sum_{k=1}^n |a_k + b_k|^p
\end{bmatrix}^\frac{1}{p}
\leq
\begin{bmatrix}
\sum_{k=1}^n |a_k|^p
\end{bmatrix}^\frac{1}{p}
+
\begin{bmatrix}
\sum_{k=1}^n |b_k|^p
\end{bmatrix}^\frac{1}{p}
\end{gather}
\newline\newline
\textbf{ANSWER:}

\textbf{Step 1:}

To establish Minkowski's inequality for sums, we'll employ Hölder's inequality. Hölder's inequality states that for any two sequences of real numbers $(x_k)$ and $(y_k)$ and any $p>1$ such that $\frac{1}{p} + \frac{1}{q} = 1$, the following holds:

$\sum_{k=1}^n |x_k y_k| \leq \left( \sum_{k=1}^n |x_k|^p \right)^{\frac{1}{p}} \left( \sum_{k=1}^n |y_k|^q \right)^{\frac{1}{q}}$
\newline\newline
\textbf{Step 2:}

Now, let's define $x_k = \frac{|a_k|^p}{\left(\sum_{k=1}^n |a_k|^p\right)^{\frac{1}{p}}}$ and $y_k = \frac{|b_k|^p}{\left(\sum_{k=1}^n |b_k|^p\right)^{\frac{1}{p}}}$. 
\newline\newline
\textbf{Step 3:}

Then, we have:

\begin{align*}
\sum_{k=1}^n |a_k + b_k|^p &= \sum_{k=1}^n |a_k + b_k|^p \cdot 1^{1-p} \\
&\leq \left( \sum_{k=1}^n |a_k|^p \right)^{\frac{1}{p}} \left( \sum_{k=1}^n |b_k|^p \right)^{\frac{1}{q}} \left( \sum_{k=1}^n 1^{q} \right)^{1/q} \\
&= \left( \sum_{k=1}^n |a_k|^p \right)^{\frac{1}{p}} \left( \sum_{k=1}^n |b_k|^p \right)^{\frac{1}{p}} \left( \sum_{k=1}^n 1 \right)^{1/q} \\
&= \left( \sum_{k=1}^n |a_k|^p \right)^{\frac{1}{p}} \left( \sum_{k=1}^n |b_k|^p \right)^{\frac{1}{p}} n^{1/q}
\end{align*}

Given $\frac{1}{p} + \frac{1}{q} = 1$, we find $q = \frac{p}{p-1}$.
\newline\newline
Thus, $n^{1/q} = n^{\frac{p-1}{p}} = n^{\frac{1}{p-1}}$.
\newline\newline
Hence:

$\sum_{k=1}^n |a_k + b_k|^p \leq \left( \sum_{k=1}^n |a_k|^p \right)^{\frac{1}{p}} \left( \sum_{k=1}^n |b_k|^p \right)^{\frac{1}{p}} n^{\frac{1}{p-1}}$
\newline\newline
\textbf{Step 4:}

Dividing both sides by $n^{\frac{1}{p-1}}$:

$\frac{1}{n^{\frac{1}{p-1}}} \sum_{k=1}^n |a_k + b_k|^p \leq \left( \sum_{k=1}^n |a_k|^p \right)^{\frac{1}{p}} \left( \sum_{k=1}^n |b_k|^p \right)^{\frac{1}{p}}$
\newline\newline\newline\newline
\textbf{Step 5:}

Taking the $p$-th root on both sides:

$\left( \frac{1}{n} \sum_{k=1}^n |a_k + b_k|^p \right)^{\frac{1}{p}} \leq \left( \sum_{k=1}^n |a_k|^p \right)^{\frac{1}{p}} \left( \sum_{k=1}^n |b_k|^p \right)^{\frac{1}{p}}$
\newline\newline
\textbf{Thus, Minkowski's inequality for sums is established.}
\newline\newline
\textbf{3. Prove the triangle inequality $|x+y| \leq |x| + |y|, \forall (x,y) \in \mathbb{R}$}
\newline\newline\newline\newline\newline\newline
\textbf{ANSWER:}

\textbf{Step 1:}

To prove the triangle inequality $|x+y| \leq |x| + |y|$, we can consider the cases where $x$ and $y$ are both positive, both negative, or of opposite signs.
\newline\newline
\textbf{Step 2:}

Case 1: $x$ and $y$ are both positive:
$|x+y| = x+y$ 
$|x| + |y| = x + y$ 
In this case, since $x$ and $y$ are both positive, their absolute values are the same as the original values. So $|x+y| = |x| + |y|$ holds true.
\newline\newline
\textbf{Step 3:}

Case 2: $x$ and $y$ are both negative:
$|x+y| = -(x+y) = -x - y$ 
$|x| + |y| = -x - y$ 
In this case, since $x$ and $y$ are both negative, their absolute values are the negations of the original values. So $|x+y| = |x| + |y|$ holds true.
\newline\newline
\textbf{Step 4:}

Case 3: $x$ and $y$ have opposite signs:
Without loss of generality, assume $x > 0$ and $y < 0$. Then,
$|x+y| = |x-y|$ 
$|x| + |y| = x + (-y) = x - y$ 
Since $x > y$ (because $x$ is positive and $y$ is negative), $|x-y| = x-y$, so $|x+y| = |x| + |y|$ holds true.
\newline\newline
\textbf{Step 5:}
\newline\newline
\textbf{In all cases, we have shown that $|x+y| \leq |x| + |y|$, thus proving the triangle inequality for all real numbers $x$ and $y$.}
\newline\newline
\textbf{4. Prove Sedrakayan's Lemma $\forall u_i, v_i \in \mathbb{R}^+$:}

\begin{gather}
\frac{(\sum_{i=1}^n u_i)^2}{\sum_{i=1}^n v_i}
\leq
\sum_{i=1}^n \frac{(u_i)^2}{v_i}
\end{gather}
\newline\newline
\textbf{ANSWER:}

\textbf{Step 1:}

We aim to prove Sedrakayan's Lemma, which asserts:

For any positive real numbers $u_i$ and $v_i$ (where $i$ ranges from 1 to $n$):
\newline\newline
$\frac{(\sum_{i=1}^n u_i)^2}{\sum_{i=1}^n v_i} \leq \sum_{i=1}^n \frac{u_i^2}{v_i}$
\newline\newline
To establish this, we'll utilize the Cauchy-Schwarz Inequality.
\newline\newline
\textbf{Step 2:}

Define two vectors:
   - $\mathbf{u} = (u_1, u_2, ..., u_n)$
   - $\mathbf{v} = (\sqrt{v_1}, \sqrt{v_2}, ..., \sqrt{v_n})$
\newline\newline
\textbf{Step 3:}

Next, we utilize the Cauchy-Schwarz Inequality for inner products of vectors. This inequality asserts that the square of the dot product of two vectors is smaller than or equal to the product of the squares of their magnitudes.
\newline\newline
\textbf{Step 4:}

For the sum of $u_i$, we have:
$(\sum_{i=1}^n u_i)^2 \leq n (\sum_{i=1}^n u_i^2)$

For the sum of $\sqrt{v_i}$, we have:
$(\sum_{i=1}^n \sqrt{v_i})^2 \leq n (\sum_{i=1}^n v_i)$
\newline\newline
\textbf{Step 5:}

Rearranging, we obtain:
$\frac{(\sum_{i=1}^n u_i)^2}{\sum_{i=1}^n v_i} \leq \frac{\sum_{i=1}^n u_i^2}{n}$
\newline\newline
\textbf{Step 6:}

Now, we know that $\mathbf{u} \cdot \mathbf{v} \leq |\mathbf{u}| \cdot |\mathbf{v}|$. Expanding the dot product:
$\sum_{i=1}^n u_i \sqrt{v_i} \leq \sqrt{\sum_{i=1}^n u_i^2} \sqrt{\sum_{i=1}^n v_i}$
\newline\newline
\textbf{Step 7:}

Square both sides:
$(\sum_{i=1}^n u_i \sqrt{v_i})^2 \leq (\sum_{i=1}^n u_i^2)(\sum_{i=1}^n v_i)$
\newline\newline
\textbf{Step 8:}

Substitute back $\mathbf{u} = (u_1, u_2, ..., u_n)$ and $\mathbf{v} = (\sqrt{v_1}, \sqrt{v_2}, ..., \sqrt{v_n})$:
$\left(\sum_{i=1}^n u_i \sqrt{v_i}\right)^2 \leq \left(\sum_{i=1}^n u_i^2\right) \left(\sum_{i=1}^n v_i\right)$
\newline\newline
\textbf{Step 9:}

Rearranging, we obtain:
$\frac{(\sum_{i=1}^n u_i)^2}{\sum_{i=1}^n v_i} \leq \sum_{i=1}^n \frac{u_i^2}{v_i}$
\newline\newline
\textbf{This validates Sedrakayan's Lemma.}

\end{flushleft}

\end{document}
